\documentclass[a4paper, 12pt]{article}
\usepackage[hmargin=1.25in,vmargin=1in]{geometry}

\usepackage{xeCJK}

%% Font
\CJKfontspec{Noto Serif CJK TC} % 思源宋體

\usepackage{parskip}
\setlength{\parindent}{2em} % 縮排兩個字
\usepackage{indentfirst}
 
\usepackage{listings} % Maybe for code block
\usepackage{underscore}

\usepackage{hyperref}
\hypersetup{
    colorlinks=true,
    linkcolor=black,
    filecolor=magenta,      
    urlcolor=cyan,
    pdftitle={Overleaf Example},
    pdfpagemode=FullScreen,
} % Copy from https://www.overleaf.com/learn/latex/Hyperlinks

% Custom varibles
\def\myProjectName{專案名稱(Project Name)}
\def\myVersion{1.0}

% Font Size
\newcommand\TwentyTitle{\fontsize{20pt}{24pt}\selectfont}
\newcommand\EighteenTitle{\fontsize{18pt}{20pt}\selectfont}
\newcommand\SixteenTitle{\fontsize{16pt}{18pt}\selectfont}
\newcommand\SectionFont{\fontsize{14pt}{16pt}\selectfont}
\newcommand\NormalFont{\fontsize{12pt}{16pt}\selectfont}

\usepackage{titlesec}
\titleformat*{\section}{\center\SectionFont\bfseries}
\titleformat*{\subsection}{\NormalFont\bfseries}
\titleformat*{\subsubsection}{\NormalFont}

%----------------------------------------------------------

\begin{document}

%titlepage
\thispagestyle{empty}
\begin{center}
    {\TwentyTitle \myProjectName \par}
    \vspace{6cm}
    {\TwentyTitle 系統需求規格書 \par}
    {\EighteenTitle Software Requirements Specification (SRS) \par}
    {\SixteenTitle Version: \myVersion \par}
    \vspace{4cm}
    {\SixteenTitle
    \renewcommand{\arraystretch}{1.3} % row height
    \begin{tabular}{ccc}
      姓名 & 學號 & E-mail \\[0.2em]
      劉建宏 & 000000000 & t000000000@ntut.org.tw \\
      劉建宏 & 000000000 & t000000000@ntut.org.tw \\
      劉建宏 & 000000000 & t000000000@ntut.org.tw \\
      劉建宏 & 000000000 & t000000000@ntut.org.tw \\
      劉建宏 & 000000000 & t000000000@ntut.org.tw \\
      劉建宏 & 000000000 & t000000000@ntut.org.tw \\
    \end{tabular}
    \renewcommand{\arraystretch}{1} % reset
    \par}
    \vspace{2cm}
    {\SixteenTitle Department of Computer Science \& Information Engineering National Taipei University of Technology \par}
    \vspace{16pt}
    {\SixteenTitle 10/11/2023 \par}
\end{center}
\clearpage

\renewcommand{\contentsname}{目錄 (Table of Contents)}
\tableofcontents
\newpage

%---------------------------------------------

% \section*{Revision History}

% \begin{center}
%     \begin{tabular}{|c|c|c|c|}
%         \hline
% 	    Name & Date & Reason For Changes & Version\\
%         \hline
% 	    21 & 22 & 23 & 24\\
%         \hline
% 	    31 & 32 & 33 & 34\\
%         \hline
%     \end{tabular}
% \end{center}

% \newpage
%----------------------------------------------

\section{簡介 (Introduction)}

\subsection{目的 (Purpose)}

藉由本學期修習的資料庫系統課程,為了更加理解資料庫的運行及流程的設計,
並且將過往學習到的網頁程式設計、網際網路技術與應用和各項程式語言的基礎學以致
用,透過分工合作完成使用資料庫的系統,因此我們決定開發一個完整的商場系統。
此系統能讓使用者可以藉由本學期修習的資料庫系統課程,為了更加理解資料庫的運行及流程的設計,
並且將過往學習到的網頁程式設計、網際網路技術與應用和各項程式語言的基礎學以致
用,透過分工合作完成使用資料庫的系統,因此我們決定開發一個完整的商場系統。

本系統主要目標為:
\begin{enumerate}
  \item 目的一
  \item 目的二
\end{enumerate}

\subsection{系統名稱 (Identification)}

藉由本學期修習的資料庫系統課程,為了更加理解資料庫的運行及流程的設計,
並且將過往學習到的網頁程式設計、網際網路技術與應用和各項程式語言的基礎學以致
用,透過分工合作完成使用資料庫的系統,因此我們決定開發一個完整的商場系統。
此系統能讓使用者可以藉由本學期修習的資料庫系統課程,為了更加理解資料庫的運行及流程的設計,
並且將過往學習到的網頁程式設計、網際網路技術與應用和各項程式語言的基礎學以致
用,透過分工合作完成使用資料庫的系統,因此我們決定開發一個完整的商場系統。

本系統主要目標為:
\begin{enumerate}
  \item 目的一
  \item 目的二
\end{enumerate}

\subsection{概觀 (Overview)}
\subsection{符號描述 (Notation Description) (if any)}

\section{系統 (System)}
\subsection{系統描述 (System Description)}
\subsubsection{系統架構圖 (System Architecture Diagram) }
\subsection{操作概念 (Operational Concepts or User Stories)}
\subsection{功能性需求 (Functional Requirements)}
\subsection{資料需求 (Data Requirements)}
\subsection{非功能性需求 (Non-Functional Requirements)}
\subsubsection{效能需求 (Performance Requirements)}
\subsubsection{資安需求 (Security Requirements) (if any)}
\subsection{介面需求 (Interface Requirements)}
\subsubsection{使用者介面需求 (User Interfaces Requirements)}
\subsubsection{外部介面需求 (External Interface Requirements) (if any)}
\subsubsection{內部介面需求 (Internal Interface Requirements) (if any)}
\subsection{其他需求 (Other Requirements)}
\subsubsection{環境需求 (Environmental Requirement)}
\subsubsection{安裝需求 (Installation Requirement)}
\subsubsection{測試需求 (Test Requirements) (if any)}
\subsection{商業規則與限制 (Business Rules and Integrity Constrains)}
\newpage

\section{Glossary}
\newpage

\section{References}
\newpage

\section{Appendix}
\newpage

\end{document}
